% latex table generated in R 4.0.1 by xtable 1.8-4 package
% Mon Aug 03 09:16:15 2020
\begin{table}[ht]
\centering
\begin{tabular}{lrrrr}
  \hline
   & \multicolumn{1}{c}{P. bassins} & \multicolumn{1}{c}{Pi�ge RG} &
   \multicolumn{1}{c}{Pi�ge RD} & \\
 \hline
  & \multicolumn{3}{c}{Anguille} & \\
 \hline
  & \multicolumn{3}{c}{Jaune} & Total\\
 \hline
1996 &  & 14 171 &  & 14 171 \\ 
  1997 &  & 8 614 &  & 8 614 \\ 
  1998 &  & 24 437 &  & 24 437 \\ 
  1999 & 2 & 33 935 &  & 33 937 \\ 
  2000 &  & 14 608 &  & 14 608 \\ 
  2001 &  & 6 336 &  & 6 336 \\ 
  2002 & 39 & 9 186 &  & 9 225 \\ 
  2003 & 121 & 9 323 &  & 9 444 \\ 
  2004 & 211 & 3 910 &  & 4 121 \\ 
  2005 &  & 878 &  & 878 \\ 
  2006 &  & 15 011 &  & 15 011 \\ 
  2007 &  & 15 255 & 414 & 15 669 \\ 
  2008 & 37 & 53 508 & 4 386 & 57 931 \\ 
  2009 & 60 & 67 109 & 4 486 & 71 656 \\ 
  2010 & 105 & 27 430 & 923 & 28 457 \\ 
  2011 & 31 & 9 127 & 188 & 9 346 \\ 
  2012 & 191 & 11 959 & 495 & 12 645 \\ 
  2013 & 111 & 119 391 & 25 601 & 145 103 \\ 
  2014 & 62 & 86 203 & 10 480 & 96 745 \\ 
  2015 & 129 & 26 454 & 3 053 & 29 636 \\ 
  2016 & -20 & 44 011 & 6 321 & 50 312 \\ 
  2017 & -118 & 21 181 & 1 901 & 22 964 \\ 
  2018 & 182 & 50 789 & 6 816 & 57 787 \\ 
  2019 & 171 & 14 811 & 2 686 & 17 668 \\ 
  2020 &  & 150 &  & 150 \\ 
   \hline
\end{tabular}
\caption{Migration des stades anguilles jaunes, dans les
		trois passes du barrage d'Arzal (P. bassin = passe � fente verticale \footnote{Les anguilles jaunes migrent principalement dans les rampes � anguilles, mais
peuvent emprunter la passe � bassins. Elles sont compt�es lorsqu'elles sont observ�es
devant la vitre de comptage mais elles peuvent passer au travers des grilles qui
orientent les poissons vers les vitres, elles sont difficilement d�tectables par
la cam�ra surtout quand elles passent au fond. Les effectifs d'anguilles
fr�quentant la passe � bassins sont juste pr�sent�s pour information.}, Pi�ge
		RG = passe pi�ge historique sur le gabion, Pi�ge RD = passe du mur guide eau. Les effectifs des op�rations � cheval sur deux ann�es sont re-
		r�partis au pro-rata des effectifs de chaque ann�e.} 
\label{table_bilanannuel_ang}
\end{table}
