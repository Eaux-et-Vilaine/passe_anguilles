% latex table generated in R 4.1.2 by xtable 1.8-4 package
% Tue Jan 04 20:18:25 2022
\begin{table}[ht]
\centering
\begin{tabular}{lrrrr}
  \hline
   & \multicolumn{1}{c}{Passe � bassins} & \multicolumn{1}{c}{Pi�ge RG} & \multicolumn{1}{c}{Pi�ge RD} & \\
 \hline
  & \multicolumn{3}{c}{Anguille} & \\
 \hline
  & \multicolumn{3}{c}{Argent�e} & Total\\
 \hline
1999 &  & 2 &  & 2 \\ 
  2000 &  & 1 &  & 1 \\ 
  2001 &  & 44 &  & 44 \\ 
  2002 &  & 3 &  & 3 \\ 
  2003 &  & 2 &  & 2 \\ 
  2004 &  & 1 &  & 1 \\ 
  2006 &  & 1 &  & 1 \\ 
  2007 &  & 1 &  & 1 \\ 
  2009 &  & 3 & 1 & 4 \\ 
  2010 &  & 6 &  & 6 \\ 
  2011 &  & 1 &  & 1 \\ 
  2012 &  & 2 &  & 2 \\ 
  2013 &  & 11 & 1 & 12 \\ 
  2015 &  & 3 &  & 3 \\ 
  2016 &  & 5 &  & 5 \\ 
  2017 &  & 4 &  & 4 \\ 
  2018 &  & 6 & 1 & 7 \\ 
  2019 & -17 & 3 &  & -14 \\ 
  2020 & -6 & 11 &  & 5 \\ 
   \hline
\end{tabular}
\caption{Migration des stades anguilles argent�es, dans les
		deux passes du barrage d'Arzal, Pi�ge RG = passe pi�ge historique
		sur le gabion, Pi�ge RD = passe du mur guide eau). Les effectifs des op�rations � cheval sur deux ann�es sont re-
		r�partis au pro-rata des effectifs de chaque ann�e. (\textit{Note : le stade des anguilles est difficile � d�terminer
		en suivi vid�o, et la plupart des anguilles migrant dans la passe � bassins
		sont class�es comme jaunes}).}
\label{table_bilanannuel_ang}
\end{table}
