% latex table generated in R 4.1.2 by xtable 1.8-4 package
% Fri Jan 14 15:46:29 2022
\begin{table}[ht]
\centering
\begin{tabular}{lrrrr}
  \hline
   & \multicolumn{1}{c}{Passe à bassins} & \multicolumn{1}{c}{Piège RG} & \multicolumn{1}{c}{Piège RD} & \\
 \hline
  & \multicolumn{3}{c}{Anguille} & \\
 \hline
  & \multicolumn{3}{c}{Argentée} & Total\\
 \hline
1999 &  & 2 &  & 2 \\ 
  2000 &  & 1 &  & 1 \\ 
  2001 &  & 44 &  & 44 \\ 
  2002 &  & 3 &  & 3 \\ 
  2003 &  & 2 &  & 2 \\ 
  2004 &  & 1 &  & 1 \\ 
  2006 &  & 1 &  & 1 \\ 
  2007 &  & 1 &  & 1 \\ 
  2009 &  & 3 & 1 & 4 \\ 
  2010 &  & 6 &  & 6 \\ 
  2011 &  & 1 &  & 1 \\ 
  2012 &  & 2 &  & 2 \\ 
  2013 &  & 11 & 1 & 12 \\ 
  2015 &  & 3 &  & 3 \\ 
  2016 &  & 5 &  & 5 \\ 
  2017 &  & 4 &  & 4 \\ 
  2018 &  & 6 & 1 & 7 \\ 
  2019 & -17 & 3 &  & -14 \\ 
  2020 & -6 & 11 &  & 5 \\ 
  2021 &  & 15 & 4 & 19 \\ 
   \hline
\end{tabular}
\caption{Migration des stades anguilles argentées, dans les
		deux passes du barrage d'Arzal, Piège RG = passe piège historique
		sur le gabion, Piège RD = passe du mur guide eau). Les effectifs des
		opérations à cheval sur deux années sont re- répartis au pro-rata des effectifs de chaque année. (\textit{Note : le stade des anguilles est difficile à déterminer
		en suivi vidéo, et la plupart des anguilles migrant dans la passe à bassins sont classées comme jaunes.})} 
\label{table_bilanannuel_ang}
\end{table}
