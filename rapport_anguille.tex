@article{beaulaton_EffectManagementMeasures_2007,
  title = {Effect of Management Measures on Glass Eel Escapement},
  author = {Beaulaton, L. and Briand, C.},
  date = {2007},
  journaltitle = {Ices Journal of Marine Science},
  shortjournal = {ICES J. Mar. Sci.},
  volume = {64},
  number = {7},
  pages = {1402--1413},
  abstract = {A model, GEMAC, glass eel model to assess compliance (GEMAC), has been developed with the objective of assessing anthropogenic impacts on glass eels in estuaries and evaluating the effects of management measures, to support initiatives aimed at helping the eel stocks recover. The model is described and applied to two estuaries with contrasting anthropogenic pressures: the Vilaine and the Garonne. It assesses the proportion of settled glass eels relative to a non-impacted situation with current (\%S/R) or pristine recruitment (\%S/R0). The estimated \%S/R (\%S/R0) is 5.5\% (1.1\%) for the Vilaine and 78\% (19\%) for the Garonne, in accord with the different levels of anthropogenic pressure in these two estuaries. A sensitivity analysis shows that the assessment of \%S/R is accurate, and that in a data-poor context, the \%S/R is under-assessed, as required by the precautionary approach. Seven management scenarios are explored all aiming to halve the anthropogenic pressure, but in fact leading to different levels of glass eel escapement, from almost zero to a 13-fold increase. This variation emphasizes the need for the estuarine context of eel stock management to be carefully evaluated for effectiveness when implementing management measures.},
  timestamp = {2017-05-28T08:58:38Z},
  file = {C\:\\Users\\cedric.briand\\Zotero\\storage\\9R6J9JID\\Beaulaton et Briand - 2007 - Effect of management measures on glass eel escapem.pdf}
}

@thesis{briand_DynamiquePopulationMigration_2009,
  title = {Dynamique de Population et de Migration Des Civelles En Estuaire de {{Vilaine}}. {{Population}} Dynamics and Migration of Glass Eels in the {{Vilaine}} Estuary},
  author = {Briand, C},
  date = {2009},
  institution = {{Agrocampus Ouest}},
  pagetotal = {207},
  file = {C\:\\Users\\cedric.briand\\OneDrive - EPTB Vilaine\\pdf\\Briand_these.pdf}
}

@article{briand_stacomir_2017,
  title = {{{StacomiR}} 0.5.1- Fish Migration Monitoring},
  author = {Briand, Cédric and Legrand, Marion and Besse, Timothée},
  date = {2017-05}
}

@report{briand_SuiviMigrationsAnguilles_2018,
  title = {Suivi Des Migrations d'anguilles Au Barrage d'{{Arzal}}},
  author = {Briand, Cédric and Sauvaget, Brice and Eriau, Gérard},
  date = {2018-06-18},
  pages = {25},
  institution = {{EPTB Vilaine}},
  url = {https://eptbvilaine56.sharepoint.com/:b:/g/extranet/EWjDphK4ou1PrwCYFnVWvKcBhSDMpEOm8Fq4P9hXLft47g?e=XBWHCf},
  abstract = {La migration sur les deux passes du barrage d'Arzal est estimée à 270 380 civelles pour un poids de 75 kg en 2017 ce qui place cette année au 13 ème rang sur 23 années de suivi. La migration sur le gabion, passe principale au centre du barrage, est estimée à 232 824 civelles soit 65 kg. Sur le mur guide eau, passe secondaire, la migration est estimée à 37 556 civelles soit 10 kg. 22 964 anguilles jaunes ont migré sur les passes, ce qui classe cette année comme la 10 ème sur 23 années de suivi. Ce rapport détaille les conditions de recrutement, capture, migration ainsi que l'évolution des caractéristiques biologiques des civelles en 2016. Il constitue la première version d'une automatisation de l'édition du rapport de suivi à l'aide des commandes du logiciel stacomiR http://stacomir.r-forge.r-project.org/.}
}

@article{desaunay_SeasonalLongtermChanges_1997,
  title = {Seasonal and Long-Term Changes in Biometrics of Eel Larvae: A Possible Relationship between Recruitment Variation and {{North Atlantic}} Ecosystems Productivity},
  author = {Désaunay, Y. and Guérault, D.},
  date = {1997},
  journaltitle = {Journal of Fish Biology},
  volume = {51},
  pages = {317--339},
  abstract = {Passage de 75 mm à 68 mm moyenne de novembre à Mars dans les années 80 à 90},
  keywords = {Anguilla anguilla,geb,growth,leptocephalus,migration,plankton,Recruitment,sea,stock},
  file = {C\:\\Users\\cedric.briand\\Zotero\\storage\\FUKKTBX3\\Desaunay-1997-biometric change recruitment eel larvae Atlantic ocean productivity.pdf}
}

@article{wood_LowrankScaleinvariantTensor_2006,
  title = {Low-Rank Scale-Invariant Tensor Product Smooths for Generalized Additive Mixed Models},
  author = {Wood, Simon N.},
  date = {2006-12},
  journaltitle = {Biometrics. Journal of the International Biometric Society},
  shortjournal = {Biometrics},
  volume = {62},
  number = {4},
  eprint = {17156276},
  eprinttype = {pmid},
  pages = {1025--1036},
  issn = {0006-341X},
  doi = {10.1111/j.1541-0420.2006.00574.x},
  abstract = {A general method for constructing low-rank tensor product smooths for use as components of generalized additive models or generalized additive mixed models is presented. A penalized regression approach is adopted in which tensor product smooths of several variables are constructed from smooths of each variable separately, these "marginal" smooths being represented using a low-rank basis with an associated quadratic wiggliness penalty. The smooths offer several advantages: (i) they have one wiggliness penalty per covariate and are hence invariant to linear rescaling of covariates, making them useful when there is no "natural" way to scale covariates relative to each other; (ii) they have a useful tuneable range of smoothness, unlike single-penalty tensor product smooths that are scale invariant; (iii) the relatively low rank of the smooths means that they are computationally efficient; (iv) the penalties on the smooths are easily interpretable in terms of function shape; (v) the smooths can be generated completely automatically from any marginal smoothing bases and associated quadratic penalties, giving the modeler considerable flexibility to choose the basis penalty combination most appropriate to each modeling task; and (vi) the smooths can easily be written as components of a standard linear or generalized linear mixed model, allowing them to be used as components of the rich family of such models implemented in standard software, and to take advantage of the efficient and stable computational methods that have been developed for such models. A small simulation study shows that the methods can compare favorably with recently developed smoothing spline ANOVA methods.},
  langid = {english},
  keywords = {Analysis of Variance,Animals,Biometry,Clinical Trials as Topic,Data Collection,Female,Fishes,Humans,Linear Models,Models,Models; Statistical,Ovum,Regression Analysis,Statistical}
}

